\chapter{Sequential and concurrent programming with Golang on PostgreSQL database retrieval.} 
\label{AppendixD} 
\lhead{Appendix D. \emph{Sequential and concurrent programming with Golang on PostgreSQL database retrieval.}}

% Write your Appendix content below here.
% =========================================

\section {Golang Sequential Program Source Code}

\lstset{basicstyle=\ttfamily\tiny}  
\begin{lstlisting}[breaklines, frame=single, numbers=left, caption={Golang Sequential Program Source Code}, label=commandline-02]

package main

	import (
	"database/sql"
	"fmt"
	"time"
	
	_ "github.com/lib/pq"
)

const (
	DB_USER     = "yinghua"
	DB_PASSWORD = "123"
	DB_NAME     = "fyp1"
)

var db *sql.DB

//====================================================
//function to check error and print error messages
//====================================================
func checkErr(err error, message string) {
	if err != nil {
		panic(message + " err: " + err.Error())
	}
}

//====================================================
// initialize connection with database
//====================================================
func initDB() {

	dbInfo := fmt.Sprintf("user=%s password=%s dbname=%s sslmode=disable",
	DB_USER, DB_PASSWORD, DB_NAME)
	psqldb, err := sql.Open("postgres", dbInfo)
	checkErr(err, "Initialize database")
	db = psqldb

}

//====================================================
// retrieve data from company table in postgres
//====================================================
func retrieveCompanyData() {

	fmt.Println("Start retrieve company data from database ... ")
	start := time.Now()
	
	time.Sleep(time.Second * 2)
	
	rows, err := db.Query("SELECT c.companyname, c.companynumber, c.companycategory, c.companystatus, c.countryoforigin FROM companydata AS c ORDER BY c.companynumber limit 100;")
	checkErr(err, "Query Company DB rows")
	
	var (
		companyname     string
		companynumber   string
		companycategory string
		companystatus   string
		countryoforigin string
	)
	
	for rows.Next() {
		err = rows.Scan(&companyname, &companynumber, &companycategory, &companystatus, &countryoforigin)
		checkErr(err, "Read company data rows")
		//fmt.Printf("%8v %3v %6v %6v %6v\n", companyname, companynumber, companycategory, companystatus, countryoforigin)
	}
	
	fmt.Println("Data retrieval of company data SUCCESS! ")
	fmt.Printf("%.8fs elapsed\n\n", time.Since(start).Seconds())

}

//====================================================
// retrieve data from postcode table in postgres
//====================================================
func retrievePostcodeData() {

	fmt.Println("Start retrieve postcode data from database ... ")
	start := time.Now()
	
	time.Sleep(time.Second * 2)
	
	rows, err := db.Query("SELECT postcode1, postcode2, date_introduce, usertype, position_quality FROM go_nspl LIMIT 50")
	checkErr(err, "Query Postcode DB rows")
	
	var (
		postcode1        string
		postcode2        string
		date_introduce   string
		usertype         int
		position_quality int
	)
	
	for rows.Next() {
		err = rows.Scan(&postcode1, &postcode2, &date_introduce, &usertype, &position_quality)
		checkErr(err, "Read postcode data rows")
		//fmt.Printf("%6v %8v %6v %6v %6v\n", postcode1, postcode2, date_introduce, usertype, position_quality)
	}
	
	fmt.Print("Data retrieval of postcode data SUCCESS! ")
	fmt.Printf("%.8fs elapsed\n\n", time.Since(start).Seconds())

}

//====================================================
// retrieve data from subject table in postgres
//====================================================
func retrieveSubjectData() {

	fmt.Println("Start retrieve LEO data from database ... ")
	start := time.Now()
	
	time.Sleep(time.Second * 2)
	
	rows, err := db.Query("SELECT ukprn, providername, region, subject, sex FROM go_subject LIMIT 50")
	checkErr(err, "Query subject DB rows")
	
	var (
		ukprn   int
		name    string
		region  string
		subject string
		sex     string
	)
	
	for rows.Next() {
		err = rows.Scan(&ukprn, &name, &region, &subject, &sex)
		checkErr(err, "Read subject data rows")
		//fmt.Printf("%6v %8v %6v %6v %6v\n", ukprn, name, region, subject, sex)
	}
	
	fmt.Print("Data retrieval of subject data SUCCESS! ")
	fmt.Printf(" %.8fs elapsed\n\n", time.Since(start).Seconds())

}

//====================================================
// Main function
//====================================================
func main() {

	// get the time before execution
	start := time.Now()
	
	initDB()
	retrieveCompanyData()
	retrievePostcodeData()
	retrieveSubjectData()
	
	// print the time after execution
	fmt.Printf("Total execution %.5fs elapsed\n", time.Since(start).Seconds())

}

/**

yinghua@yinghua:~/Desktop/apps/eclipse-workspace/FYP1/src/postgres-process$ go build sequential-psql.go
yinghua@yinghua:~/Desktop/apps/eclipse-workspace/FYP1/src/postgres-process$ time go run sequential-psql.go
Start retrieve company data from database ...
Data retrieval of company data SUCCESS!
2.00721985s elapsed

Start retrieve postcode data from database ...
Data retrieval of postcode data SUCCESS!
2.00144933s elapsed

Start retrieve LEO data from database ...
Data retrieval of subject data SUCCESS!
2.00131415s elapsed

Total execution 6.01005s elapsed

real	0m6.252s
user	0m0.272s
sys		0m0.032s


**/


\end{lstlisting}

\pagebreak

\subsection {Golang Concurrent Program Source Code}

\lstset{basicstyle=\ttfamily\tiny} 
\begin{lstlisting}[breaklines, frame=single, numbers=left, caption={Golang Concurrent Program Source Code}, label=commandline-02]

package main

import (
	"database/sql"
	"fmt"
	"time"
	
	_ "github.com/lib/pq"
)

//====================================================
// database information
//====================================================
const (
	DB_USER     = "yinghua"
	DB_PASSWORD = "123"
	DB_NAME     = "fyp1"
)

var (
	db          *sql.DB
	numChannels int = 3
)

//====================================================
// function to check error and print error messages
//====================================================
func checkErr(err error, message string) {
	if err != nil {
		panic(message + " err: " + err.Error())
	}
}

//====================================================
// initialize connection with database
//====================================================
func initDB() {

	dbInfo := fmt.Sprintf("user=%s password=%s dbname=%s sslmode=disable",
	DB_USER, DB_PASSWORD, DB_NAME)
	psqldb, err := sql.Open("postgres", dbInfo)
	checkErr(err, "Initialize database")
	db = psqldb

}

//====================================================
// retrieve company data store in postgres database
//====================================================
func retrieveCompanyData(ch_company chan string) {

	fmt.Println("Start retrieve company data from database ... ")
	start := time.Now()
	
	time.Sleep(time.Second * 2)
	
	rows, err := db.Query("SELECT c.companyname, c.companynumber, c.companycategory, c.companystatus, c.countryoforigin FROM companydata AS c ORDER BY c.companynumber limit 100;")
	checkErr(err, "Query Company DB rows")
	
	var (
		companyname     string
		companynumber   string
		companycategory string
		companystatus   string
		countryoforigin string
	)
	
	for rows.Next() {
		err = rows.Scan(&companyname, &companynumber, &companycategory, &companystatus, &countryoforigin)
		checkErr(err, "Read company data rows")
		//fmt.Printf("%8v %3v %6v %6v %6v\n", companyname, companynumber, companycategory, companystatus, countryoforigin)
	}
	
	fmt.Printf("%.8fs elapsed\n", time.Since(start).Seconds())
	ch_company <- "Retrieval of company data success. \n"
}

//====================================================
// retrieve postcode data store in postgres database
//====================================================
func retrievePostcodeData(ch_postcode chan string) {
	
	fmt.Println("Start retrieve postcode data from database ... ")
	start := time.Now()
	
	time.Sleep(time.Second * 2)
	
	rows, err := db.Query("SELECT postcode1, postcode2, date_introduce, usertype, position_quality FROM go_nspl LIMIT 50")
	checkErr(err, "Query Postcode DB rows")
	
	var (
		postcode1        string
		postcode2        string
		date_introduce   string
		usertype         int
		position_quality int
	)
	
	for rows.Next() {
		err = rows.Scan(&postcode1, &postcode2, &date_introduce, &usertype, &position_quality)
		checkErr(err, "Read postcode data rows")
		//fmt.Printf("%6v %8v %6v %6v %6v\n", postcode1, postcode2, date_introduce, usertype, position_quality)
	}
	
	fmt.Printf("%.8fs elapsed\n", time.Since(start).Seconds())
		ch_postcode <- "Retrieval of postcode success. \n"
	}
	
	//====================================================
	// retrieve subject data store in postgres database
	//====================================================
	func retrieveSubjectData(ch_subject chan string) {
	
		fmt.Println("Start retrieve LEO data from database ... ")
		start := time.Now()
		
		time.Sleep(time.Second * 2)
		
		rows, err := db.Query("SELECT ukprn, providername, region, subject, sex FROM go_subject LIMIT 50")
		checkErr(err, "Query subject DB rows")
		
		var (
			ukprn   int
			name    string
			region  string
			subject string
			sex     string
		)
		
		for rows.Next() {
			err = rows.Scan(&ukprn, &name, &region, &subject, &sex)
			checkErr(err, "Read subject data rows")
			//fmt.Printf("%6v %8v %6v %6v %6v\n", ukprn, name, region, subject, sex)
		}
		
		fmt.Printf("%.8fs elapsed\n", time.Since(start).Seconds())
			ch_subject <- "Retrieval of subject data success. \n"
		}
		
		// select function
		func goSelect(ch_company, ch_subject, ch_postcode chan string) {
		
		for i := 0; i < numChannels; i++ {
		
			select {
			case msg1 := <-ch_postcode:
				fmt.Println(msg1)
			case msg2 := <-ch_company:
				fmt.Println(msg2)
			case msg3 := <-ch_subject:
				fmt.Println(msg3)
		
		}
	
	}
}

//====================================================
// Main function
//====================================================
func main() {
	
	// make three channel for three functions
	ch_company := make(chan string)
	ch_subject := make(chan string)
	ch_postcode := make(chan string)
	
	// get the time before execution
	start := time.Now()
	
	initDB()
	
	//go routines
	go retrieveCompanyData(ch_company)
	go retrieveSubjectData(ch_subject)
	go retrievePostcodeData(ch_postcode)
	
	goSelect(ch_company, ch_subject, ch_postcode)
	
	// obtain the time after execution
	fmt.Printf("Total execution %.5fs elapsed\n", time.Since(start).Seconds())

}

/**

yinghua@yinghua:~/Desktop/apps/eclipse-workspace/FYP1/src/postgres-process$ go build concurrent-psql.go
yinghua@yinghua:~/Desktop/apps/eclipse-workspace/FYP1/src/postgres-process$ time go run concurrent-psql.go
Start retrieve postcode data from database ...
Start retrieve company data from database ...
Start retrieve LEO data from database ...
2.00615007s elapsed
Retrieval of subject data success.

2.00661550s elapsed
Retrieval of postcode success.

2.00745319s elapsed
Retrieval of company data success.

Total execution 2.00754s elapsed

real	0m2.268s
user	0m0.244s
sys		0m0.076s



**/

)

\end{lstlisting}

