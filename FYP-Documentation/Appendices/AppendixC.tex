\chapter{Golang programming for import CSV into PostgreSQL database} 
\label{AppendixC} 
\lhead{Appendix C. \emph{Golang for import CSV into PostgreSQL database.}} 

% Write your Appendix content below here.
% =========================================

\section{Introduction}

The Go Programming Language possess package csv to reads and write comma-separated values (CSV) files. The package will automatically ignore whitespace, blank lines and delimits commas to read data. In addition, the language also contains a driver to perform CRUD transaction on PostgreSQL database. 

The program below imports 100 rows of company data, LEO data and NSPL data from CSV files to PostgreSQL database. Five columns of data are selected from each file to import into this program as proof of concept in this project. The tables will be created in PostgreSQL database before the program is executed.

\pagebreak

\subsection {LEO table for data importation}

\lstset{basicstyle=\ttfamily\tiny} 
\begin{lstlisting}[breaklines, frame=single, numbers=left, caption={PostgreSQL query for LEO table creation.}, label=commandline-02]

-- File: fyp1-leo.sql
-- Author: Chai Ying Hua 
-- Database: psql (PostgreSQL) 9.5.8

-- ======================================
-- CHANGES IN V1.1(Sun Aug 27. 2017)
-- 	Create leo table for phase 1 to import data
-- ======================================

create table go_subject  ( 
	ukprn int not null, 
	providername varchar(100) not null,
	region varchar(100) not null, 
	subject varchar(50) not null, 
	sex varchar(30) not null
);

\end{lstlisting}

\subsection {NSPL table for data importation}

\lstset{basicstyle=\ttfamily\tiny} 
\begin{lstlisting}[breaklines, frame=single, numbers=left, caption={PostgreSQL query for NSPL table creation.}, label=commandline-02]

-- File: fyp1-nspl.sql
-- Author: Chai Ying Hua 
-- Database: psql (PostgreSQL) 9.5.8

-- ======================================
-- CHANGES IN V1.1(Mon Sep 4. 2017)
-- 	Create nspl table for phase 1 to import data
-- ======================================

create table go_nspl (
	postcode1 varchar(15) not null,
	postcode2 varchar(15) not null primary key,
	date_introduce varchar(10) not null, 
	usertype int not null,
	position_quality int not null
)

\end{lstlisting}

\subsection {LEO table for data importation}

\lstset{basicstyle=\ttfamily\tiny} 
\begin{lstlisting}[breaklines, frame=single, numbers=left, caption={PostgreSQL query for Company table creation.}, label=commandline-02]

-- File: fyp1-company.sql
-- Author: Chai Ying Hua 
-- Database: psql (PostgreSQL) 9.5.8

-- =========================================================================
-- CHANGES IN V1.1(Sun Aug 27. 2017)
-- 	Create companydata table for phase 1 to import data
-- =========================================================================

create table go_company ( 
	CompanyName varchar(160) null default null, 
	CompanyNumber varchar(8) not null primary key,
	CompanyCategory varchar(100) not null,
	CompanyStatus varchar(70) not null
	CountryOfOrigin varchar(50) not null
);

\end{lstlisting}



\subsection{Source code of Go program}

\lstset{basicstyle=\ttfamily\tiny} 
\begin{lstlisting}[breaklines, frame=single, numbers=left, caption={Source code of Go program}, label=commandline-02]

package main

import (
	"bufio"
	"database/sql"
	"encoding/csv"
	"fmt"
	"io"
	"os"
	"strconv"
	
	_ "github.com/lib/pq"
)

const (
	DB_USER                       = "yinghua"
	DB_PASSWORD                   = "123"
	DB_NAME                       = "fyp1"
	COMPANY_FILE_DIRECTORY string = "/home/yinghua/Documents/FYP-data/company-data/company-data-full.csv"
	LEO_FILE_DIRECTORY     string = "/home/yinghua/Documents/FYP-data/subject-data/institution-subject-data.csv"
	NSPL_FILE_DIRECTORY    string = "/home/yinghua/Documents/FYP-data/postcode-data/UK-NSPL.csv"
)

type CompanyData struct {
	name     string
	number   string
	category string
	status   string
	country  string
}

type LEOData struct {
	ukprn   int
	name    string
	region  string
	subject string
	sex     string
}

type NSPLData struct {
	postcode1      string
	postcode2      string
	date_introduce string
	usertype       int
	pos_quality    int
}

var db *sql.DB

//====================================================
//function to check error and print error messages
//====================================================
func checkErr(err error, message string) {
	if err != nil {
		panic(message + " err: " + err.Error())
	}
}

//====================================================
// initialize connection to database
//====================================================
func initDB() {

	dbInfo := fmt.Sprintf("user=%s password=%s dbname=%s sslmode=disable",
	DB_USER, DB_PASSWORD, DB_NAME)
	psqldb, err := sql.Open("postgres", dbInfo)
	checkErr(err, "psql open")
	db = psqldb

}

//====================================================
// Import company data
//====================================================
func importCompanyData() {

	var sStmt string = "insert into go_company values ($1, $2, $3, $4, $5)"
	
	stmt, err := db.Prepare(sStmt)
	checkErr(err, "Prepare Stmt")
	
	// Open CSV files
	csvFile, err := os.Open(COMPANY_FILE_DIRECTORY)
	checkErr(err, "Open CSV")
	
	defer csvFile.Close()
	
	// Create a new reader.
	reader := csv.NewReader(bufio.NewReader(csvFile))
	
	for i := 0; i <= 100; i++ {
		record, err := reader.Read()
		
		// skipped the first line
		if i == 0 {
			continue
		}
		
		// Stop at EOF.
		if err == io.EOF {
			break
		}
		
		company := CompanyData{
			name:     record[0],
			number:   record[1],
			category: record[10],
			status:   record[11],
			country:  record[12],
		}
		
		stmt.Exec(company.name, company.number, company.category, company.status, company.country)
		checkErr(err, "Company Data importation")
	}
}

//====================================================
// Import LEO data
//====================================================
func importSubjectData() {

	var sStmt string = "insert into go_subject values ($1, $2, $3, $4, $5)"
	
	stmt, err := db.Prepare(sStmt)
	checkErr(err, "Prepare Subject Stmt")
	
	csvFile, err := os.Open(LEO_FILE_DIRECTORY)
	checkErr(err, "Open LEO CSV")
	
	defer csvFile.Close()
	
	// Create a new reader.
	reader := csv.NewReader(bufio.NewReader(csvFile))
	
	for i := 0; i <= 100; i++ {
		record, err := reader.Read()
		
		// skipped the first line
		if i == 0 {
			continue
		}
		
		// Stop at EOF.
		if err == io.EOF {
			break
		}
		
		integer, err := strconv.Atoi(record[0])
		checkErr(err, "Convert UKRPN to Integer")
		
		subject := LEOData{
			ukprn:   integer,
			name:    record[1],
			region:  record[2],
			subject: record[3],
			sex:     record[4],
		}
		
			stmt.Exec(subject.ukprn, subject.name, subject.region, subject.subject, subject.sex)
			checkErr(err, "Subject Data importation")
		}
}

//====================================================
// Import NSPL data
//====================================================
func importNSPLData() {

	var sStmt string = "insert into go_nspl values ($1, $2, $3, $4, $5)"
	
	stmt, err := db.Prepare(sStmt)
	checkErr(err, "Prepare Postcode Stmt")
	
	csvFile, err := os.Open(NSPL_FILE_DIRECTORY)
	checkErr(err, "Open Postcode CSV")
	
	defer csvFile.Close()
	
	// Create a new reader.
	reader := csv.NewReader(bufio.NewReader(csvFile))

	for i := 0; i <= 100; i++ {
		record, err := reader.Read()
		
		// skipped the first line
		if i == 0 {
		continue
		}
		
		// Stop at EOF.
		if err == io.EOF {
		break
		}
		
		userInt, err := strconv.Atoi(record[4])
		checkErr(err, "Convert Usertype to Integer")
		
		posInt, err := strconv.Atoi(record[7])
		checkErr(err, "Convert Usertype to Integer")
	
		postcode := NSPLData {
		postcode1:      record[0],
		postcode2:      record[1],
		date_introduce: record[3],
		usertype:       userInt,
		pos_quality:    posInt,
		}
	
		stmt.Exec(postcode.postcode1, postcode.postcode2, postcode.date_introduce, postcode.usertype, postcode.pos_quality)
		checkErr(err, "Postcode Data importation")
	}
}

func main() {

	initDB()
	importCompanyData()
	importSubjectData()
	importNSPLData()

}

/**

yinghua@yinghua:~/Desktop/apps/eclipse-workspace/FYP1/src/postgres-process$ go build import-csv-psql.go
yinghua@yinghua:~/Desktop/apps/eclipse-workspace/FYP1/src/postgres-process$ time go run import-csv-psql.go

real	0m3.647s
user	0m0.328s
sys	0m0.096s
yinghua@yinghua:~/Desktop/apps/eclipse-workspace/FYP1/src/postgres-process$

**/

\end{lstlisting}
