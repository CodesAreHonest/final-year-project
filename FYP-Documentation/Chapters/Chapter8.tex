\chapter{Conclusions} 
% Main chapter title

\label{Chapter8} 
%Call reference to this chapter use \ref{ChapterX}

\lhead{Chapter 8. \emph{Conclusions}} 
% Change X to a consecutive number; this is for the header on each page - perhaps a shortened title

\doublespacing
% LINE FORMATTING

%\clearpage
%\pagebreak

% MAIN SECTION ==============================
\section{Conclusions}

In phase 1, we have review many concepts and addressed the details of concurrent programming language concepts.

The project objectives for Phase 1 are:
\begin{enumerate}[topsep=0pt,itemsep=-1ex,partopsep=1ex,parsep=1.5ex]
	
	\item To learn and understand about Go and RUST programming language concepts and their concurrent processing features.  
	\item To conduct a comparison on Go programming language concepts in processing big data with different techniques. 
	\item To implement the handling of big data with PostgreSQL, an object-oriented relational database management system (OORDBMS)
	
\end{enumerate}

\pagebreak

What we have achieved on Phase 1:
\begin{enumerate}[topsep=0pt,itemsep=-1ex,partopsep=1ex,parsep=1.5ex]
	
	\item We reviewed different concepts and characteristics of concurrent programming language.
	\item We established the fundamentals of concurrent programming knowledge and possess confident advance to the next phase of development.
	\item We established a development platform for concurrency programming.
	\item We demonstrated the capability of concurrent programming language, which is provide better performance and throughput on data processing compare to sequential programming with results.
	
\end{enumerate}


%The project objectives for Phase 2 are:
%\begin{enumerate}[topsep=0pt,itemsep=-1ex,partopsep=1ex,parsep=1.5ex]
%	
%	\item To develop a G-code interpreter program on RTAI Linux.
%	\item To develop an FPGA-based CNC interpolation driver to control Panasonic AC Servo system.
%	\item To produce a closed-loop control system for the CNC controller.
%\end{enumerate}
%
%What we have achieved on Phase 2:
%\begin{enumerate}[topsep=0pt,itemsep=-1ex,partopsep=1ex,parsep=1.5ex]
%	\item We have implemented a Linux program that can interpret G-Codes into signal commands that can be sent to FPGA for synchronized interpolation
%	\item We have implemented a Linux program that can send signal commands from PC to FPGA and receive an acknowledgment from FPGA to PC.    
%	\item We have implemented an FPGA-based CNC Drive that can receive signal commands from PC and generate multi-axis digital signals to drive the CNC.
%	\item We did not manage to develop an entire closed-loop control system, however, we manage to establish fundamental designs for a closed-loop control system by extending the current implementation.
%\end{enumerate}


\section{Lessons Learned}

\begin{enumerate}[topsep=0pt,itemsep=-1ex,partopsep=1ex,parsep=1.5ex]
	\item \textbf{Data science knowledge.} Data science is being use as competitive weapon and it transform the way how companies operate with information. It is a totally new knowledge and experience for me as Software Engineering student to learn and explore.
	
	\item \textbf{Concurrent programming concepts.} Concurrent concepts is difficult to be understand and never thought in subject syllabus. Learning the art of concurrent programming for building applications in this project provide satisfaction and motivation to fulfill my desire to build a real-time system. 
	
	\item \textbf{Consistent update with FYP Supervisor.} FYP supervisor ensure the project is on track and doing right. It is essential to make available time for consultation and rapidly update the progress for supervisor via email to enhance the work quality. Moreover, FYP supervisor review my work ensure the time and resource is not waste on doing the wrong task. 	 
	
	\item \textbf{Ubuntu Operating System.} The project allow me to learn Linux Bash commands through practice. I had found Ubuntu operating system is not hard to use and its more safety, reliable and consistent to conduct development activities due to its lightweight.
	
\end{enumerate}

\section{Recommendations for Future Work}

\begin{enumerate}[topsep=0pt,itemsep=-1ex,partopsep=1ex,parsep=1.5ex]
	\item \textbf{GORM for CRUD on data processing.} GORM is an Object-relational mapping (ORM) library for Golang that converting data from incompatible files types into struct or interface. For instance, this project does not use GORM to import data and possess poor readability, error handling and maintainability in program. It is recommend to import data with GORM package because it supports auto migration, associations with database and every features are tested. 
	\linebreak
	
	\item \textbf{Benchmark on language performance comparsion.} Although this project possess well-defined of benchmarking on database table spacing, hardware configuration and amount of query execution on data retrieval to conduct language performance comparison. These benchmarks are insufficient to determine the accurateness of programming language performance. This is because the CPU usage might be running on other processes or program while conducting the performance test. It is recommend to unified number of processes running in background and programming style for performance comparison between different concurrent programming languages.
	
	\item \textbf{Data quality.} Although this project use Devil Advocation Test method to identify data quality. The method is insufficient to ensure data obtained is valid, complete and accurate to be processed. It is recommend to use several scripting language such as Python and Perl to identify internal data consistency and data cleansing is required to eliminate duplication of data. 
	
\end{enumerate}









